% 第4章 公開と運用
\chapter{公開と運用}

\section{サービスリリース}

2025年12月12日に本作品をリリースし\cite{polyseek},以後X(旧Twitter)にて継続的な広報活動を行っている.また,Discordサーバを開設し,ユーザからのフィードバックを収集している.

\section{実装済み機能}

リリース時点で,認証・ユーザ機能,商品データ機能,タグシステム,検索機能,管理機能の5つの主要機能が動作している.以下,各機能について詳述する.

\subsection{認証・ユーザ機能}

第2章で述べたOAuth 2.0認証により,GoogleアカウントまたはDiscordアカウントによるログインを提供している.ログイン後のユーザには,プロフィール設定,「いいね」した商品や「所有済み」商品の一覧管理機能が利用可能となる.

\subsection{商品データ機能}

商品データの収集には,BOOTHからの情報スクレイピング機能を実装している.ユーザが商品URLを入力すると,システムは自動的に商品タイトル,価格,画像URL,説明文などの情報を取得し,データベースに登録する.既に登録済みの商品については,最新情報への自動更新も行われる.

商品詳細ページでは,取得した商品情報に加え,ユーザコミュニティによって付与されたタグや,他のユーザからの評価情報を確認できる.

\begin{figure}[H]
    \centering
    \includegraphics[width=\textwidth]{figures/product_detail.pdf}
    \caption{商品詳細ページ}
    \label{fig:product_detail}
\end{figure}

\subsection{タグシステム}

第2章で設計したタグシステムが稼働している.ログイン済みのユーザは,任意の商品に対してタグを自由に追加・削除でき,すべての編集操作はバージョン管理される.他のユーザは編集履歴を閲覧し,各編集に対して賛成または反対の評価を投票できる.

\subsection{検索機能}

第3章で述べた検索システムにより,キーワード検索,タグ・カテゴリによる絞り込み,価格帯フィルタリング,および複合条件検索が利用可能である.

\begin{figure}[H]
    \centering
    \includegraphics[width=\textwidth]{figures/search_result.pdf}
    \caption{検索結果ページ}
    \label{fig:search_result}
\end{figure}

\subsection{管理機能}

管理者向けには,タグ管理画面を提供している.この画面では,タグの統合や削除,カテゴリの設定などの操作が可能である.

\begin{figure}[H]
    \centering
    \includegraphics[width=\textwidth]{figures/admin_panel_1.pdf}
    \caption{管理者向けタグ一覧画面}
    \label{fig:admin_panel_1}
\end{figure}

\begin{figure}[H]
    \centering
    \includegraphics[width=\textwidth]{figures/admin_panel_2.pdf}
    \caption{タグ詳細編集画面}
    \label{fig:admin_panel_2}
\end{figure}

\begin{figure}[H]
    \centering
    \includegraphics[width=\textwidth]{figures/sentry_dashboard.pdf}
    \caption{Sentryエラー監視ダッシュボード}
    \label{fig:sentry_dashboard}
\end{figure}

\section{利用状況}

\subsection{現時点での実績}

2025年1月12日時点での利用状況について述べる.登録ユーザ数は9人であり,ユーザによって作成されたタグ数は92個に達している.サービス公開から約1ヶ月という期間を考慮すると,タグデータベースの構築は順調に進んでいると評価できる.

本作品の価値は,ユーザコミュニティによって蓄積されるタグデータにある.現時点では少人数での運用であるが,92個のタグが作成されていることから,アクティブユーザ1人あたり平均10個以上のタグを作成している計算になる.これは,ユーザがタグ付け機能を積極的に活用していることを示している.

\subsection{今後の成長に向けて}

タグベースの検索システムは,タグデータが増えるほど検索の精度と網羅性が向上するネットワーク効果を持つ.現在はまだ初期段階であるが,ユーザ数とタグ数が増加することで,BOOTHの公式検索では実現できない詳細な検索が可能になると期待される.

\section{運用対応}

\subsection{フィードバックに基づく改善}

サービス公開後は,継続的な運用対応を行っている.Discordサーバを通じてユーザからのフィードバックを収集し,バグ報告や機能改善の要望に対応している.

これまでに対応した主な改善点として,検索結果の表示速度の改善や,タグ入力時のオートコンプリート機能の追加がある.これらの改善は,ユーザからの具体的なフィードバックに基づいて優先順位を決定し,実装を行った.

\subsection{サーバ監視と安定運用}

本番環境の安定運用のため,サーバの監視とパフォーマンス最適化にも取り組んでいる.第2章で述べたSentryによるエラー監視やPM2によるプロセス管理により,問題の早期発見とダウンタイムの最小化を実現している.
