% 第1章 作品概要
\chapter{作品概要}

\section{サービス概要}

本作品「PolySeek」は,BOOTH\cite{booth}で販売される3Dアバターや衣装等のデジタルアセットを対象とした検索Webサービスである.ユーザは商品URLを入力するだけで,タイトルや画像等の情報をデータベースに自動登録できる.

最大の特徴は,ユーザコミュニティによる自由な「タグ付け」機能である.「青いリボン」「レース生地」といった公式カテゴリにはない詳細な特徴での検索を可能にし,膨大な商品から理想のアセットを効率的に発見できる環境を提供する.


\begin{figure}[H]
    \centering
    \includegraphics[width=0.9\textwidth]{figures/top_page.pdf}
    \caption{PolySeekトップページ}
    \label{fig:top_page}
\end{figure}

\section{対象ユーザ}

本作品の対象者はVRChatなどのメタバースプラットフォームで活動するクリエイタや一般ユーザである.特に,BOOTHで3Dアバター,衣装等のデジタルアセットを頻繁に購入する人々を主な対象とする.

\section{解決する課題}

\subsection{市場の急成長と検索の困難化}

VRChat内で使用できるアバターやアセットの多くは,日本のクリエイタ向けマーケットプレイスであるBOOTH\cite{booth}で取引されている.VRChatのユーザ増加に合わせてBOOTHの利用ユーザも増加し,2024年度の3Dモデルカテゴリの注文件数は約402万件を超えている\cite{booth_whitepaper}.これは前年比で101\%の注文件数の増加である.取引高に関しても2024年度は58億円,前年比187\%増という記録的な増加が見られている.

BOOTHはVRChatと提携しており,VRChatカテゴリには多数の商品が出品されている.BOOTHを利用して商品を購入するユーザが増えるにつれ,BOOTHへVRChat向けアイテムを出品するクリエイタの数も増加している.

\subsection{既存検索機能の限界}

BOOTHの取引高が増える一方で,3Dモデルカテゴリの商品数増加に伴い,ユーザが「欲しい」と思う商品を効率的に見つけ出すことが難しくなってきている.現在のBOOTHでは商品に対してタグを付与することができるが,このタグは出品者のみが編集可能である.そのため,出品者によって商品に付けるタグの個数にばらつきがあり,「フリル」「テーパード」といったデザインに関するファッション用語のタグの活用が限定的であったり,「サイバーパンク」「ファンタジー」といった商品が属するジャンルのタグの定義が曖昧な状態となっている.

現在のBOOTHの検索機能はキーワード検索が中心であり,膨大な商品の中から目的のものを探し出すには多くの時間と労力がかかっている.例えば,検索欄に「VRChat」「こたつ」というキーワードを入力した場合,検索結果にはこたつそのもののアセットだけでなく,説明文に「こたつ」という単語が含まれている衣装やアセットも表示されてしまう.さらに,商品の並び順は初期状態で「人気順」となっているが,この人気順の基準が不明瞭であり,既に人気の商品だけが注目されやすいという問題もある.

\subsection{本作品によるアプローチ}

こうした課題を解決するためには,販売者が提供する公式情報だけに頼るのではなく,ユーザが持つ商品知識を検索に活用できる仕組みが必要である.本作品では,BOOTHの商品に対してアプリケーション内で誰でもタグを付与・削除できるシステムを構築し,ユーザコミュニティ主導のデータベースを作成することで,検索体験の向上を図る.

\newpage
\section{制作目的}

本作品では,以下の三点を目的として開発を行った.

\begin{enumerate}
    \item \textbf{ユーザコミュニティによるタグ付けシステムの導入}\\
    より実用的で質の高いデジタルアセットのデータベースを構築する.

    \item \textbf{モダンなユーザインタフェースの提供}\\
    よく使う検索条件や機能に容易にアクセスできるようにすることで検索体験そのものを改善する.

    \item \textbf{商品の一元管理機能}\\
    ユーザが関心を持った商品や購入した商品をプラットフォーム上で一元的に管理できる機能を提供する.
\end{enumerate}

これらにより,情報収集やアセット管理の効率を高め,上述の課題解決を目指している.
