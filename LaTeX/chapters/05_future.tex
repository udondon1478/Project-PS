% 第5章 今後の予定
\chapter{今後の予定}

今後は,より快適にサービスを利用してもらうための機能実装およびプロモーションに注力する.本章では,短期的・中期的・継続的なタスクに分類して,今後の開発計画を述べる.

\section{短期的なタスク}

短期的には,既存の設計を活かした機能拡張と,サービスの認知度向上に取り組む.

\subsection{含意関係モデルの活用}

データベーススキーマで定義済みの含意関係モデルを活用し,よりインテリジェントなタグ登録機能の実装を目指す.具体的には,「パーカー」というタグを登録した際に,上位タグである「トップス」も自動登録される機能を実装する.

この機能により,ユーザのタグ付け作業が簡略化されるとともに,検索時のヒット率が向上する.また,データベース内のタグに一貫性が生まれ,検索品質の向上が期待できる.

\subsection{パフォーマンス最適化}

ユーザ数およびデータ量の増加に備え,パフォーマンス最適化に取り組む.データベースクエリの効率化やキャッシュ戦略の改善により,応答速度の維持を図る.また,画像配信についてはCDNの活用を検討し,ユーザ体験の向上を目指す.

\subsection{プロモーション活動}

サービスの認知度向上のため,SNSを活用したプロモーション活動を継続する.X(旧Twitter)での定期的な機能紹介や開発状況の報告を通じて,潜在的なユーザへのリーチを図る.また,VRChat関連コミュニティへの広報活動や,既存ユーザからのフィードバック収集も並行して進める.

\section{中期的なタスク}

中期的には,ユーザビリティの向上と国際化対応に注力する.

\subsection{UI/UXの改善}

ユーザフィードバックに基づき,インターフェースの継続的な改善を行う.特に,モバイル端末での利用体験の向上は優先度が高い.また,検索フィルターのUI改善やアクセシビリティの向上にも取り組み,より多くのユーザが快適に利用できる環境を整備する.

\subsection{多言語対応の基盤整備}

開発初期から海外展開を想定し,データベーススキーマにタグ翻訳モデルを用意している.この設計を活かし,日本語タグに対する英語翻訳の提案機能を実装する.また,ユーザの言語設定に応じたUI切り替え機能や,翻訳コミュニティの構築にも取り組み,国際的なユーザベースの拡大を目指す.

\section{継続的なタスク}

サービス運用においては,いくつかの対応を継続的に行う必要がある.商品出品者からの掲載取り下げ依頼への対応は,権利者の意向を尊重する上で重要である.また,ユーザからのフィードバックを反映した機能改善や,スパムや不正なタグ編集を行う悪質なユーザへの対応も欠かせない.

技術面では,セキュリティアップデートの適用やサーバ監視・障害対応を継続し,安定したサービス運用を維持する.
